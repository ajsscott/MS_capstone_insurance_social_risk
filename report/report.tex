% Options for packages loaded elsewhere
% Options for packages loaded elsewhere
\PassOptionsToPackage{unicode}{hyperref}
\PassOptionsToPackage{hyphens}{url}
\PassOptionsToPackage{dvipsnames,svgnames,x11names}{xcolor}
%
\documentclass[
  number,
  review,
  3p]{elsarticle}
\usepackage{xcolor}
\usepackage{amsmath,amssymb}
\setcounter{secnumdepth}{5}
\usepackage{iftex}
\ifPDFTeX
  \usepackage[T1]{fontenc}
  \usepackage[utf8]{inputenc}
  \usepackage{textcomp} % provide euro and other symbols
\else % if luatex or xetex
  \usepackage{unicode-math} % this also loads fontspec
  \defaultfontfeatures{Scale=MatchLowercase}
  \defaultfontfeatures[\rmfamily]{Ligatures=TeX,Scale=1}
\fi
\usepackage{lmodern}
\ifPDFTeX\else
  % xetex/luatex font selection
\fi
% Use upquote if available, for straight quotes in verbatim environments
\IfFileExists{upquote.sty}{\usepackage{upquote}}{}
\IfFileExists{microtype.sty}{% use microtype if available
  \usepackage[]{microtype}
  \UseMicrotypeSet[protrusion]{basicmath} % disable protrusion for tt fonts
}{}
\makeatletter
\@ifundefined{KOMAClassName}{% if non-KOMA class
  \IfFileExists{parskip.sty}{%
    \usepackage{parskip}
  }{% else
    \setlength{\parindent}{0pt}
    \setlength{\parskip}{6pt plus 2pt minus 1pt}}
}{% if KOMA class
  \KOMAoptions{parskip=half}}
\makeatother
% Make \paragraph and \subparagraph free-standing
\makeatletter
\ifx\paragraph\undefined\else
  \let\oldparagraph\paragraph
  \renewcommand{\paragraph}{
    \@ifstar
      \xxxParagraphStar
      \xxxParagraphNoStar
  }
  \newcommand{\xxxParagraphStar}[1]{\oldparagraph*{#1}\mbox{}}
  \newcommand{\xxxParagraphNoStar}[1]{\oldparagraph{#1}\mbox{}}
\fi
\ifx\subparagraph\undefined\else
  \let\oldsubparagraph\subparagraph
  \renewcommand{\subparagraph}{
    \@ifstar
      \xxxSubParagraphStar
      \xxxSubParagraphNoStar
  }
  \newcommand{\xxxSubParagraphStar}[1]{\oldsubparagraph*{#1}\mbox{}}
  \newcommand{\xxxSubParagraphNoStar}[1]{\oldsubparagraph{#1}\mbox{}}
\fi
\makeatother


\usepackage{longtable,booktabs,array}
\usepackage{calc} % for calculating minipage widths
% Correct order of tables after \paragraph or \subparagraph
\usepackage{etoolbox}
\makeatletter
\patchcmd\longtable{\par}{\if@noskipsec\mbox{}\fi\par}{}{}
\makeatother
% Allow footnotes in longtable head/foot
\IfFileExists{footnotehyper.sty}{\usepackage{footnotehyper}}{\usepackage{footnote}}
\makesavenoteenv{longtable}
\usepackage{graphicx}
\makeatletter
\newsavebox\pandoc@box
\newcommand*\pandocbounded[1]{% scales image to fit in text height/width
  \sbox\pandoc@box{#1}%
  \Gscale@div\@tempa{\textheight}{\dimexpr\ht\pandoc@box+\dp\pandoc@box\relax}%
  \Gscale@div\@tempb{\linewidth}{\wd\pandoc@box}%
  \ifdim\@tempb\p@<\@tempa\p@\let\@tempa\@tempb\fi% select the smaller of both
  \ifdim\@tempa\p@<\p@\scalebox{\@tempa}{\usebox\pandoc@box}%
  \else\usebox{\pandoc@box}%
  \fi%
}
% Set default figure placement to htbp
\def\fps@figure{htbp}
\makeatother





\setlength{\emergencystretch}{3em} % prevent overfull lines

\providecommand{\tightlist}{%
  \setlength{\itemsep}{0pt}\setlength{\parskip}{0pt}}



 
\usepackage[]{natbib}
\bibliographystyle{elsarticle-harv}


\setcitestyle{authoryear,open={(},close={)}}
\makeatletter
\@ifpackageloaded{caption}{}{\usepackage{caption}}
\AtBeginDocument{%
\ifdefined\contentsname
  \renewcommand*\contentsname{Table of contents}
\else
  \newcommand\contentsname{Table of contents}
\fi
\ifdefined\listfigurename
  \renewcommand*\listfigurename{List of Figures}
\else
  \newcommand\listfigurename{List of Figures}
\fi
\ifdefined\listtablename
  \renewcommand*\listtablename{List of Tables}
\else
  \newcommand\listtablename{List of Tables}
\fi
\ifdefined\figurename
  \renewcommand*\figurename{Figure}
\else
  \newcommand\figurename{Figure}
\fi
\ifdefined\tablename
  \renewcommand*\tablename{Table}
\else
  \newcommand\tablename{Table}
\fi
}
\@ifpackageloaded{float}{}{\usepackage{float}}
\floatstyle{ruled}
\@ifundefined{c@chapter}{\newfloat{codelisting}{h}{lop}}{\newfloat{codelisting}{h}{lop}[chapter]}
\floatname{codelisting}{Listing}
\newcommand*\listoflistings{\listof{codelisting}{List of Listings}}
\makeatother
\makeatletter
\makeatother
\makeatletter
\@ifpackageloaded{caption}{}{\usepackage{caption}}
\@ifpackageloaded{subcaption}{}{\usepackage{subcaption}}
\makeatother
\journal{elsarticle}
\usepackage{bookmark}
\IfFileExists{xurl.sty}{\usepackage{xurl}}{} % add URL line breaks if available
\urlstyle{same}
\hypersetup{
  pdftitle={Predicting Auto Insurance Risk Using Gradient Boosting},
  pdfauthor={AJ Strauman-Scott},
  pdfkeywords={Gradient Boosting, XGBoost, SHAP
explainability, hyperparameter optimization, auto insurance
risk, American Community Survey (ACS), NYC Open Data, predictive
modeling, socio-economic predictors, crash modeling},
  colorlinks=true,
  linkcolor={blue},
  filecolor={Maroon},
  citecolor={Blue},
  urlcolor={Blue},
  pdfcreator={LaTeX via pandoc}}


\setlength{\parindent}{6pt}
\begin{document}

\begin{frontmatter}
\title{Predicting Auto Insurance Risk Using Gradient
Boosting \\\large{Analyzing Socio-Economic Predictors in Crash Data for
New York City} }
\author[1]{AJ Strauman-Scott%
%
}
 \ead{true} 

\affiliation[1]{organization={City University Of New York
(CUNY), Department of Data Science},city={New York City},country={United
States of America},countrysep={,},postcodesep={}}

\cortext[cor1]{Corresponding author}

        
\begin{abstract}
This study explores the use of the gradient boosting model XGBoost to
predict auto insurance risk by integrating socio-economic variables from
publicly available data. By treating crash frequency as proxy for
insurance claims, the project aims to identify key neighborhood-level
factors influencing risk. The dataset, encompassing 13,518 tract-year
observations from 2018 to 2023, captures demographic, economic, housing,
and commuting indicators alongside engineered interaction variables.
Optuna hyperparameter tuning and SHAP-based explainability reveal that
post-pandemic traffic dynamics, median gross rent, percent of a
population in the labor-force, and the interaction of poverty with
vehicle ownership are significant predictors of crash risk. While the
model achieves moderate predictive accuracy (\(R^2\) = 0.26), its
interpretability highlights socio-economic disparities that influence
urban traffic safety. The findings underscore the potential of open
data-driven models for portfolio-level risk assessment and urban safety
planning, while cautioning against direct use for individual
underwriting due to fairness and legal concerns. Future work should
incorporate telematics data and fairness-aware algorithms to improve
granularity and reduce bias.
\end{abstract}





\begin{keyword}
    Gradient Boosting \sep XGBoost \sep SHAP
explainability \sep hyperparameter optimization \sep auto insurance
risk \sep American Community Survey (ACS) \sep NYC Open
Data \sep predictive modeling \sep socio-economic predictors \sep 
    crash modeling
\end{keyword}
\end{frontmatter}
    

\section{Introduction}\label{introduction}

Accurate insurance risk modeling is critical for setting fair premiums,
mitigating losses, and ensuring financial stability within the insurance
industry \citep{henckaerts, clemente}. Predicting claim frequency and
severity not only supports pricing but also enables insurers to manage
portfolio-level risk and optimize resource allocation \citep{mohamed}.

New York City (NYC) presents a complex urban environment where traffic
risks are shaped by socio-economic factors, dense infrastructure, and
scaling dynamics typical of large metropolitan areas
\citep{cabrera, bettencourt}. The availability of open datasets---such
as NYC's Motor Vehicle Collision (MVC) data and socio-economic
indicators from the American Community Survey (ACS)---offers a unique
opportunity to develop proxy models for insurance claim risk. These data
sources provide detailed insights into crash frequency, injury severity,
commuting behaviors, and neighborhood-level demographics
\citep{adeniyi, brubacher}.

Traditional actuarial methods, such as Generalized Linear Models (GLMs),
have long been the foundation of risk pricing and underwriting due to
their interpretability and regulatory acceptance \citep{henckaerts}.
However, GLMs are limited in their ability to capture non-linear
relationships and interactions among complex predictors like
socio-economic factors, urban infrastructure, and driving behavior
\citep{clemente}. These limitations are particularly pronounced in urban
contexts, where crash risk is shaped by heterogeneous population
dynamics and localized factors \citep{cabrera, brubacher}.

There is a growing need for data-driven approaches that can flexibly
incorporate diverse predictors while addressing the complex temporal and
spatial patterns of accidents highlighted in recent reviews
\citep{grigorev, behboudi}. Recent studies and systematic reviews
confirm that machine learning methods, particularly ensemble models like
Gradient Boosting Machines (GBMs), outperform traditional GLMs for
predicting both claim frequency and severity
\citep{clemente, mohamed, behboudi}. These models are capable of
handling mixed data types (categorical and continuous) and capturing
complex feature interactions that linear models often miss.

To address the interpretability challenge of ``black box'' ML models,
SHAP (SHapley Additive exPlanations) offers a principled framework for
feature attribution, allowing insurers and policymakers to understand
both global feature importance and instance-level predictions
\citep{lundberg, dong, ning}. This combination of high-performance
prediction and explainability provides a strong foundation for modern
risk modeling \citep{kim}.

Despite the growing body of work applying GBMs to insurance modeling,
few studies integrate publicly available crash data with socio-economic
indicators to model claim-related risks specifically for the automotive
insurance sector. Most research remains limited to proprietary
policyholder data \citep{henckaerts, mohamed}, while systematic reviews
highlight that few studies combine open crash data with socio-economic
indicators in insurance modeling \citep{ali, behboudi}.

This study aims to integrate ACS socio-economic features with NYC MVC
crash data to develop an explainable gradient boosting framework for
measureing social risk for automotive insurance and urban policy. The
ultimate goal is to identify key socio-economic and transportation
predictors that drive claim frequency.

The remainder of this paper is organized as follows: Section 2 reviews
prior work on ML in insurance risk modeling, crash and socio-economic
data, geospatial analytics, model explainability, and literature gaps;
Section 3 details the data sources, key metrics, modeling approach, and
SHAP-based explainability; Section 4 reports the results including
hyperparameter tuning results, model performance, and feature
importance; Section 5 discusses the findings in relation to existing
research and industry applications; and Section 6 concludes with key
contributions, limitations, and directions for future research.

\section{Related Work}\label{related-work}

\subsection{\texorpdfstring{\textbf{Machine Learning in Insurance Risk
Modeling}}{Machine Learning in Insurance Risk Modeling}}\label{machine-learning-in-insurance-risk-modeling}

The transition from traditional actuarial models such as Generalized
Linear Models (GLMs) to machine learning (ML) approaches has marked a
significant evolution in insurance risk modeling. GLMs have historically
served as the backbone for pricing and claim prediction due to their
interpretability and regulatory acceptance. However, they are limited by
their linearity and inability to naturally capture complex interactions
and nonlinear relationships among predictors, such as driver
demographics, vehicle characteristics, socio-economic factors, and
driving behavior. As \citet{clemente} note, while GLMs remain effective
for modeling claim severity with smaller and noisier datasets, they
often underperform compared to ensemble methods when modeling claim
frequency, where nonlinearities and heterogeneous risk patterns are
prevalent. Similarly, \citet{jonkheijm} demonstrated that tree-based
models, especially XGBoost \citep{xgboost}, substantially improved
predictive accuracy over linear regression, particularly when
incorporating both actuarial features (e.g., policyholder age, vehicle
value) and behavioral indicators.

Recent studies have validated the predictive superiority of ML
methods---such as random forests, GBMs, and neural networks---over
traditional actuarial models. GBMs, such as XGBoost and LightGBM, have
emerged as particularly effective tools in auto insurance risk modeling
\citep{henckaerts}. Their iterative boosting framework enables them to
handle mixed data types (categorical and continuous) and capture
intricate patterns that GLMs and single decision trees may miss.
\citet{clemente} applied gradient boosting to both claim frequency and
severity modeling, demonstrating significant performance gains in
frequency prediction over Poisson-based GLMs. Similarly,
\citet{jonkheijm} employed XGBoost for forecasting individual claim
amounts, outperforming both regression trees and random forests.

\subsection{\texorpdfstring{\textbf{Use of Crash and Socio-Economic
Data}}{Use of Crash and Socio-Economic Data}}\label{use-of-crash-and-socio-economic-data}

Crash data has been widely recognized as a reliable proxy for insurance
claim frequency, given the direct link between the occurrence of traffic
accidents and subsequent claims filed by policyholders. Studies
utilizing police crash reports, telematics, and open transportation
datasets consistently demonstrate strong correlations between crash
frequency and insurance risk metrics \citep{takale}. The integration of
socio-economic features---including income levels, commuting patterns,
vehicle ownership rates, and population density---has been shown to
enhance the explanatory power of crash and claim prediction models.

For example, \citet{adeniyi} utilized a decade of NYC crash data
(2013--2023) to identify key predictors of accident severity---such as
unsafe speed, alcohol involvement, and adverse weather---which align
closely with the variables insurers use to model claim likelihood.
Similarly, \citet{dong} applied boosting-based ensemble models to
traffic injury severity prediction, finding that vehicle type, collision
mode, and environmental conditions strongly influenced both injury
outcomes and, by extension, potential claim costs. \citet{brubacher}
conducted a geospatial analysis of 10 years of crashes in British
Columbia and found that regions with lower income and higher
socio-economic deprivation exhibited higher rates of pedestrian crashes,
severe injuries, and fatalities, reflecting disparities in road safety
linked to infrastructure quality and enforcement intensity.
\citet{cabrera} expanded on this by identifying superlinear scaling of
road accidents in urban areas, where higher population densities led to
disproportionate increases in crash frequency, especially for minor
collisions. These findings are directly relevant for insurers, as they
imply that socio-economic and urban structural factors---such as
commuting patterns or access to public transit---can serve as proxies
for underlying risk exposure.

Urban-focused studies have further illuminated the unique risk dynamics
in metropolitan environments like New York City, Chicago, and London,
where complex traffic patterns, dense road networks, and high pedestrian
activity elevate accident risk. \citet{adeniyi} analyzed NYC crash data
to show how the COVID-19 pandemic altered accident patterns, with fewer
total crashes but an increase in injury severity due to higher vehicle
speeds on less congested roads. \citet{feng}, studying UK traffic data,
emphasized the value of big data platforms and spatial clustering
techniques (e.g., accident hotspot detection) to identify urban risk
zones, a concept that parallels insurer efforts to assess region-based
risk for underwriting.

Collectively, these studies support the notion that combining crash data
with socio-economic indicators offers a powerful means of modeling
insurance claim frequency.

\subsection{\texorpdfstring{\textbf{Explainability in GBM
Models}}{Explainability in GBM Models}}\label{explainability-in-gbm-models}

In high-stakes fields such as insurance pricing, underwriting, and
claims management, the interpretability of ML models is not only a
technical preference but also a regulatory and business requirement.
Insurers must be able to justify rating factors and risk scores to
regulators, policyholders, and internal stakeholders. Traditional
actuarial models like GLMs are naturally interpretable due to their
linear structure and explicit coefficient estimates. However, modern ML
models---such as gradient boosting or neural networks---are often
criticized as ``black boxes,'' complicating the explanation of
predictions that influence financial decisions or customer premiums.
Regulatory frameworks, including the EU's General Data Protection
Regulation (GDPR) and U.S. state-level insurance guidelines,
increasingly require transparency in algorithmic decision-making,
further amplifying the need for explainable AI (shortened to XAI).
\citet{henckaerts} further underscore this, showing that variable
importance plots and PDPs can yield actionable insights into driver and
policyholder risk factors, blending predictive power with
interpretability.

Among XAI methods, SHAP (SHapley Additive exPlanations) has become the
state-of-the-art framework for interpreting complex ML models. Developed
by \citet{lundberg}, SHAP is grounded in cooperative game theory,
assigning each feature a Shapley value that quantifies its contribution
to individual predictions. Unlike traditional feature importance
metrics---such as Gini importance in random forests or split gain in
XGBoost---SHAP accounts for both main effects and feature interactions,
offering a consistent and additive explanation of how variables drive
model outputs.

Tools like SHAP allow practitioners to interpret complex models by
quantifying the contribution of each variable to the predictions.
Studies like \citet{mohamed} highlight the value of such
interpretability when using gradient boosting for pricing and fraud
detection, as insurers must justify rating factors for regulatory
compliance.

In the insurance domain, SHAP has been widely applied to interpret
models for claims prediction, fraud detection, and risk scoring.
\citet{dong} used SHAP in conjunction with boosting-based models
(LightGBM and CatBoost) to analyze the contribution of driver age,
vehicle type, and collision type to injury severity predictions,
providing insights that aligned with domain expertise. Similarly,
\citet{ning} demonstrated how Shapley Variable Importance Cloud
(ShapleyVIC) builds on SHAP principles to assess variable significance
with uncertainty intervals, enabling fairer and more transparent risk
predictions. These approaches not only improve trust in ML-driven
decision-making but also help insurers identify the most actionable risk
factors influencing claims.

\subsection{\texorpdfstring{\textbf{Gaps in the
Literature}}{Gaps in the Literature}}\label{gaps-in-the-literature}

While ML methods---particularly ensemble models like gradient
boosting---have gained traction in insurance risk modeling, there is a
notable absence of studies that combine socio-economic and crash data
for claim risk prediction. Most existing research focuses on proprietary
insurance datasets containing policyholder and vehicle information
\citep{clemente, henckaerts, jonkheijm}. However, publicly available
crash datasets, such as NYC's Motor Vehicle Collision (MVC) reports, and
socio-economic features from the American Community Survey (ACS) remain
underutilized in insurance modeling. This gap limits the development of
robust, regionally sensitive models that capture the real-world
interaction between crash frequency) and socio-economic indicators. By
integrating geospatially-coded ACS data with MVC records, it becomes
possible to construct granular, location-aware risk models that better
reflect variations in driving exposure, infrastructure quality, and
neighborhood-level risk factors.

\section{Materials and Methods}\label{materials-and-methods}

\subsection{\texorpdfstring{\textbf{Data
Sources}}{Data Sources}}\label{data-sources}

This study integrates publicly available crash data from New York City
with socio-economic features from the American Community Survey (ACS) to
develop a proxy model for insurance claim risk. The data sources and
preprocessing steps are designed to replicate key factors used in
actuarial risk models while incorporating broader socio-economic and
regional variables.

\subsubsection{\texorpdfstring{\textbf{Crash Data (Claim
Proxies)}}{Crash Data (Claim Proxies)}}\label{crash-data-claim-proxies}

Crash data is obtained from the NYC Motor Vehicle Collisions (MVC) Open
Data Portal, covering the years 2018--2023. Each record includes details
such as crash location, number of injuries and fatalities, vehicle type,
and contributing factors (e.g., driver behavior, environmental
conditions). These variables are well-documented predictors of both
accident severity and insurance claims \citep{adeniyi, dong}.

Crash frequency was aggregated at the 2020 census tract level and
normalized by tract-level population to compute crashes per 1,000
resident. This metric will replace claim frequency \citep{brubacher}.

\subsubsection{\texorpdfstring{\textbf{Socio-Economic Data (ACS
Features)}}{Socio-Economic Data (ACS Features)}}\label{socio-economic-data-acs-features}

Socio-economic variables are drawn from the ACS 5-year estimates
(2018--2023) at the 2020 census tract level. The variables include
demographic composition (e.g., \% male, \% white, \% Hispanic, \%
foreign-born), age distribution, and income indicators. Additional
features include median gross rent, housing tenure, educational
attainment, employment metrics, and transportation factor. Interaction
features poverty × vehicle ownership and unemployment × vehicle
ownership were engineered to capture compound effects on risk exposure.

\begin{longtable}[]{@{}
  >{\raggedright\arraybackslash}p{(\linewidth - 4\tabcolsep) * \real{0.0526}}
  >{\raggedright\arraybackslash}p{(\linewidth - 4\tabcolsep) * \real{0.1805}}
  >{\raggedright\arraybackslash}p{(\linewidth - 4\tabcolsep) * \real{0.7669}}@{}}
\caption{ACS tables used and derived variables.}\tabularnewline
\toprule\noalign{}
\begin{minipage}[b]{\linewidth}\raggedright
ACS
\end{minipage} & \begin{minipage}[b]{\linewidth}\raggedright
Description
\end{minipage} & \begin{minipage}[b]{\linewidth}\raggedright
Derived.Variables
\end{minipage} \\
\midrule\noalign{}
\endfirsthead
\toprule\noalign{}
\begin{minipage}[b]{\linewidth}\raggedright
ACS
\end{minipage} & \begin{minipage}[b]{\linewidth}\raggedright
Description
\end{minipage} & \begin{minipage}[b]{\linewidth}\raggedright
Derived.Variables
\end{minipage} \\
\midrule\noalign{}
\endhead
\bottomrule\noalign{}
\endlastfoot
B01001 & Age and Sex & total\_population, male\_population,
female\_population, age\_under\_18, age\_18\_34, age\_35\_64,
age\_65\_plus \\
B01003 & Total Population & total\_population \\
B08301 & Transportation to Work & drive\_alone, carpool,
public\_transit, walk, bike, work\_from\_home \\
B08303 & Travel Time to Work & commute\_short, commute\_medium,
commute\_long \\
B19001 & Household Income & income\_under\_25k, income\_25k\_75k,
income\_75k\_plus, median\_income \\
B25010 & Average Household Size & average\_household\_size \\
B25044 & Vehicles Available & no\_vehicle, one\_vehicle,
two\_plus\_vehicles \\
C24010 & Occupation & occupation variables (aggregated) \\
C24030 & Industry & industry variables (aggregated) \\
B15003 & Education & less\_than\_hs, hs\_diploma, some\_college,
associates\_degree, bachelors\_degree, graduate\_degree \\
B17001 & Poverty Status & below\_poverty, above\_poverty,
poverty\_rate \\
B02001 & Race & white\_population, black\_population,
asian\_population \\
B03002 & Hispanic or Latino & hispanic\_population \\
B18101 & Disability Status & disability variables (aggregated) \\
B16005 & Language Spoken at Home & foreign\_born \\
B23025 & Employment Status & in\_labor\_force, employed, unemployed,
not\_in\_labor\_force, unemployment\_rate \\
B25064 & Median Gross Rent & median\_gross\_rent \\
\end{longtable}

\subsubsection{\texorpdfstring{\textbf{Preprocessing}}{Preprocessing}}\label{preprocessing}

Crash records from the NYC Open Data MVC dataset are cleaned (removing
rows with missing or zero coordinates) and spatially joined to 2020
Census Tracts using official census tract shapefiles. Annual summaries
of total crashes, injuries, and fatalities are then aggregated by tract
and normalized by tract-level population to compute per-capita crash,
injury, and fatality rates. The ACS socio-economic data are harmonized
to 2020 tract boundaries (via crosswalks for 2018--2019), binned into
interpretable categories, and converted to percentages of total
population where applicable. Interaction features---such as poverty ×
vehicle ownership and unemployment × vehicle ownership---are engineered
to capture compounded socio-economic risk factors.

After examination, a subset of highly correlated variables were removed.
Measures of poverty level and population above the poverty line, as well
as employment and unemployment percentages, were closely tied to broader
income and labor force indicators already included in the model,
creating redundancy without improving predictive power. Similarly,
metrics describing commuting alone by car and the distribution of
vehicle ownership (households with no vehicles, one vehicle, or multiple
vehicles) were binned to prevent issues from being strongly
interrelated. The precentage of female share of the population was
excluded because of its near-perfect correlation with the male share of
the population. The same was true for the share of high-income
households (earning above \$75,000), which closely overlapped with
median income levels.

No categorical encoding besides \texttt{year} or standardization was
performed at this stage since all ACS features are already expressed as
continuous percentages or numeric values, and gradient boosting models
(XGBoost) handle raw scales effectively \citep{henckaerts}.

\subsection{\texorpdfstring{\textbf{Key
Metrics}}{Key Metrics}}\label{key-metrics}

The primary risk metric, \texttt{crash\_rate\_per\_1000}, measures the
number of crashes per 1,000 residents in each census tract by year. This
population-adjusted rate follows the methodology of studies that
normalize crash counts by population to ensure fair comparisons of
relative risk across areas with varying exposure levels
\citep{brubacher, cabrera}.

This response variableis modeled alongside the socio-economic and
transportation variables detailed in Table 1. Together, these variables
allow the model to capture both the exposure risk (frequency) and
potential cost severity of accidents, aligning with the frameworks used
in both insurance \citep{clemente, henckaerts} and traffic safety
research \citep{dong}.

\subsection{\texorpdfstring{\textbf{Modeling
Approach}}{Modeling Approach}}\label{modeling-approach}

To model the relationship between socio-economic characteristics and
crash risk, we implemented a single gradient boosting framework using
XGBoost.

XGBoost \citep{xgboost} is chosen for its strong track record in
insurance risk modeling and interpretability when combined with SHAP
\citep{dong}. This model selection aligns with studies comparing
boosting frameworks for both frequency-severity modeling
\citep{henckaerts} and urban crash prediction \citep{adeniyi}.

Model performance was optimized with hypermater tuning using the
automated Bayesian optimization framework Optuna \citep{optuna}. This
approach is supported by prior research showing that systematic
hyperparameter optimization significantly improves boosting model
accuracy \citep{liu}.

Each configuration of Optuna tuning was evaluated using spatial
cross-validation at the borough level on the training data to balance
bias and variance, ensuring that the model captured meaningful patterns
without overfitting or overgeneralizing across geography. This approach
was sufficient for our dataset size and feature set, yielding robust
improvements in predictive accuracy.

\subsubsection{Explainability}\label{explainability}

Given the regulatory and operational need for transparent, explainable
models in insurance \citep{henckaerts, lundberg}, we employ SHAP
(SHapley Additive exPlanations) for both global and local feature
analysis. SHAP values are aggregated across the dataset to quantify
overall feature importance, revealing which socio-economic and
crash-related variables most influence predicted claim frequency and
severity.

\section{Results}\label{results}

\subsection{\texorpdfstring{\textbf{Descriptive
Statistics}}{Descriptive Statistics}}\label{descriptive-statistics}

The dataset comprises 13,518 census tract--year observations from 2018
to 2023. Population counts vary widely across tracts, with a median of
approximately 42,979 residents and extremes ranging from fewer than 100
to over 220,000. Demographically, the average tract population is
predominantly White (mean 3.17\%), Hispanic (mean 2.21\%), and Black
(mean 2.02\%), though these percentages vary considerably across the
five boroughs.

\begin{longtable}[]{@{}llll@{}}
\caption{Summary Statistics of Key Variables (Avg Across Tracts and
Years)}\tabularnewline
\toprule\noalign{}
Variable & Mean & SD & Max \\
\midrule\noalign{}
\endfirsthead
\toprule\noalign{}
Variable & Mean & SD & Max \\
\midrule\noalign{}
\endhead
\bottomrule\noalign{}
\endlastfoot
Crash Rate per 1,000 People & 1.43 & 1.83 & 23.29 \\
Median Gross Rent & \$1,613.24 & \$623.12 & \$3,501.00 \\
Median Annual Income & \$74,681.04 & \$40,140.23 & \$250,001.00 \\
Percent Below Poverty & 1.36\% & 1.05\% & 11.11\% \\
Percent Commute Public Transit & 3.72\% & 1.41\% & 12.31\% \\
Percent Own a Vehicle & 1.48\% & 0.66\% & 7.69\% \\
\end{longtable}

Crash-related statistics show an average of 1.43 crashes per 1,000
residents, with the highest-risk tracts exceeding 23 crashes per 1,000
residents---nearly 20 times the citywide mean. Commuting behaviors show
a multi-modal distribution, with public transit usage averaging 3.7\%,
but certain tracts---particularly in Manhattan---exhibit much higher
pedestrian and transit activity.

Economic indicators reveal stark disparities. The median household
income across all tracts and years is \$74,681, but with a standard
deviation of \$40,140 and a maximum of \$250,001, indicating substantial
variation in neighborhood wealth. Housing costs are also highly
variable, with a median gross rent of \$1,589 and a maximum of \$3,501.

\begin{figure}[H]

{\centering \pandocbounded{\includegraphics[keepaspectratio]{plots/crash_rate_by_borough_plot.png}}

}

\caption{Crash Rate by Borough, by Year}

\end{figure}%

Crash Rate by Borough shows a clear downward trend in crash rates across
all boroughs over time, with a sharp reduction during the COVID-19
pandemic period (2020) and gradual stabilization afterward. Bronx and
Queens consistently show higher crash rates per 1,000 residents compared
to Staten Island and Manhattan.

\begin{figure}[H]

{\centering \pandocbounded{\includegraphics[keepaspectratio]{plots/2018_2022_comparison.png}}

}

\caption{Crash Rate in 2018 vs 2022}

\end{figure}%

The geospatial heatmaps illustrate how crash rates are spatially
clustered within the city. In 2018, high crash rates were concentrated
in central Brooklyn, the South Bronx, and sections of northern
Manhattan, while in 2022, these hotspots persisted but appeared less
intense overall, consistent with the downward temporal trend
post-Covid-19 pandemic. By applying the same color scale across both
maps, it is evident that most census tracts saw a reduction in crash
intensity, although isolated high-risk corridors remain.

\subsection{\texorpdfstring{\textbf{Hyperperameter
Tuning}}{Hyperperameter Tuning}}\label{hyperperameter-tuning}

Optuna hyperparameter tuning \citep{optuna} enhanced the predictive
accuracy and generalization capability of the XGBoost model. The final
configuration represents a balance between model complexity and
overfitting risk, as determined by performance on the training and
validation subsets.

\begin{longtable}[]{@{}lll@{}}
\caption{Optimal Parameters as per Optuna}\tabularnewline
\toprule\noalign{}
Action & Parameter & Value \\
\midrule\noalign{}
\endfirsthead
\toprule\noalign{}
Action & Parameter & Value \\
\midrule\noalign{}
\endhead
\bottomrule\noalign{}
\endlastfoot
Learning Rate & \texttt{eta} & 1.549545 \\
Tree Depth & \texttt{max\_depth} & 4 \\
Row Sampling & \texttt{subsample} & 0.5583975 \\
Feature Sampling & \texttt{colsample\_bytree} & 0.8797697 \\
Minimum Child Weight & \texttt{min\_child\_weight} & 3 \\
Minimum Loss Reduction & \texttt{gamma} & 3.35839 \\
L2 Regularization & \texttt{lambda} & 6.131609 \\
L1 Regularization & \texttt{alpha} & 2.805485 \\
\end{longtable}

A tree depth of 4 means the model only splits features a few times
before making predictions. The socio-economic and crash-rate features
capture patterns that can be learned with relatively few decision
boundaries.

This high learning rate is unusually high compared to common XGBoost
defaults (0.01--0.3), indicating the model is taking larger steps while
fitting each tree. The relatively small dataset (13,518 observations)
allows a faster convergence without needing a very gradual learning
rate. However, high \texttt{eta} combined with shallow trees requires
careful monitoring for overfitting, which the regularization terms help
mitigate.

The subsampling ratios (\texttt{subsample\ =\ 0.56},
\texttt{colsample\_bytree\ =\ 0.88}) show the model uses just over half
of the rows for each boosting iteration and nearly all features. This
randomness prevents any one subset of data from dominating the model and
improves generalization. This confirms the dataset has sufficient size
and diversity to benefit from row sampling, but not so many redundant
features that aggressive column sampling is needed.

Regularization parameters (\texttt{lambda\ =\ 6.13},
\texttt{alpha\ =\ 2.81}, \texttt{gamma\ =\ 3}.36) are relatively strong
regularization values, indicating the model needed constraints to avoid
overfitting. In practice, this means there is enough correlation and
redundancy among features that despite removing highy correlated
variables, the model benefits from being penalized for complex splits or
large feature weights.

\subsection{\texorpdfstring{\textbf{Model Performance and
Diagnostics}}{Model Performance and Diagnostics}}\label{model-performance-and-diagnostics}

The XGBoost model achieved robust predictive performance on the holdout
test set, with the following metrics:

\begin{longtable}[]{@{}ll@{}}
\caption{XGBOOST Model Evaluation Metrics}\tabularnewline
\toprule\noalign{}
Metric & Score \\
\midrule\noalign{}
\endfirsthead
\toprule\noalign{}
Metric & Score \\
\midrule\noalign{}
\endhead
\bottomrule\noalign{}
\endlastfoot
RMSE & 1.549545 \\
MAE & 0.8372956 \\
\(R^2\) & 0.2595503 \\
\end{longtable}

The XGBoost model's evaluation metrics suggest moderate predictive power
with stable error bounds on the holdout test set. An RMSE of 1.55
indicates that the model's predicted crash rates deviate, on average, by
roughly 1.5 crashes per 1,000 residents from observed values, with an
MAE of 0.84 confirming that most prediction errors are below 1 crash per
1,000 residents. While the \(R^2=0.26\) shows that the model explains
only about 26\% of the variance in crash rates across census tracts,
this is consistent with the high degree of randomness and unobserved
factors (e.g., driver behavior, weather) inherent in crash data.

\begin{figure}[H]

{\centering \pandocbounded{\includegraphics[keepaspectratio]{plots/residuals_diagnostic_grid.png}}

}

\caption{Residual Diagnostic Plots}

\end{figure}%

\subsubsection{\texorpdfstring{\textbf{Examination of the
Residuals}}{Examination of the Residuals}}\label{examination-of-the-residuals}

The predicted vs.~actual values plot shows that the model captures the
general trend of observed crash rates across census tracts, with most
predictions clustering closely around the 45-degree reference line.
While a slight underestimation is evident for the highest crash-rate
tracts, this is typical for gradient boosting models where extreme
outliers are smoothed during ensemble averaging.

The residual density plot indicates residuals are centered near zero
with a narrow peak, suggesting minimal systemic bias.

The residuals vs.~predicted values plot shows a random scatter around
zero, with no strong patterns of heteroskedasticity or underfitting,
although a few outlier tracts exhibit residuals above ±10.

\subsection{\texorpdfstring{\textbf{Global Feature Importance
(SHAP)}}{Global Feature Importance (SHAP)}}\label{global-feature-importance-shap}

\begin{figure}[H]

{\centering \pandocbounded{\includegraphics[keepaspectratio]{plots/pdp/pdp_grid.png}}

}

\caption{Global Feature Importance with SHAP}

\end{figure}%

The SHAP plot highlights the most influential features on crash risk
predictions. The top six variables, in order of importance, are:

\textbf{Post-Pandemic Indicator}

The pre vs post Covid pandemic indicator variable hows a marked upward
shift in predicted crash rates from 2020 onward, consistent with
observed pandemic-era changes in traffic dynamics, also observed by
\citet{adeniyi}.

\textbf{Aging Population}

The PDP plot suggests that areas with a moderate share of elderly
residents (10--20\%) have slightly lower crash risks, potentially due to
lower driving exposure. However, beyond 20\%, predicted crash risk
rises.

\textbf{High School Graduate Population}

For percent of the population with a high school diploma (but no further
education), the PDP exhibits a U-shaped relationship: tracts with either
very low or very high shares of residents holding a high school diploma
are associated with higher crash risk.

\textbf{Median Gross Rent}

Median gross rent displays a positive gradient: tracts with higher
rents---often denser and more urbanized---show higher crash rates.

\textbf{Working Population}

The PDP for percentage of the population in the workforce indicates that
crash risk peaks around 60--70\% labor force participation. Lower
participation areas may have fewer commuters, while areas with higher
participation rates may experience higher traffic volumes.

\textbf{Poverty Rate x Vehicle Ownership Interaction}

Th variable marking the interaction of vehicle ownership with poverty
rate shows that high poverty rates combined with high vehicle ownership
strongly elevate crash risk. This finding suggests that economically
vulnerable drivers may face both infrastructure and behavioral risks.

\begin{figure}[H]

{\centering \pandocbounded{\includegraphics[keepaspectratio]{plots/multi_tree_plot.png}}

}

\caption{Figure X: Multi-Tree Plot}

\end{figure}%

The multi-tree plot provides a detailed view of how the XGBoost model
partitions the socio-economic and transportation variables to predict
crash risk, revealing intricate interactions that are not visible in
single-variable importance metrics. For insurers and policymakers, this
tree-based view offers actionable insights into which combinations of
conditions---such as low-income, high-commute areas with dense walking
activity---are most predictive of risk.

At the top of the tree hierarchy, the model consistently splits on the
post-pandemic indicator. Subsequent splits emphasize variables such as
age distribution, labor force participation, and foreign-born
population, all of which indicate that neighborhoods with specific
demographic compositions face distinct patterns of risk exposure. For
instance, branches that combine higher shares of elderly residents with
elevated public transit use or lower workforce participation show
heightened predicted crash rates.

Another prominent theme in the tree structure is the role of housing and
economic indicators, such as median gross rent and income brackets,
which appear alongside transportation variables. The multi-tree pathways
involving poverty rate and vehicle ownership interaction are
particularly illuminating, as they reveal how economic stress combined
with high vehicle reliance amplifies crash likelihood. For example,
nodes where the model isolates high poverty rates, high vehicle
ownership, and long commutes lead to leaves with some of the highest
predicted crash values.

The branching structure also highlights the interplay between
transportation modes and demographics. Variables like walking, biking,
and working from home repeatedly appear as secondary splits, indicating
that these behaviors act as modifiers of the primary socio-economic risk
factors. Areas with high pedestrian activity combined with mid-range
incomes or high percentages of young adults tend to diverge from the
patterns observed in primarily car-dependent neighborhoods. This
reflects the unique urban mobility landscape of New York City, where
mixed transportation modes and densely populated corridors create more
opportunities for vehicle-pedestrian conflicts.

\section{Discussion}\label{discussion}

Although the model's \(R^2\) value of approximately 0.26 indicates that
it explains just over one-quarter of the variance in crash rates across
New York City census tracts, this is a meaningful achievement
considering the inherent randomness of crash events and the absence of
individual-level data on drivers, vehicles, road conditions, weather,
road design, etc. Auto collisions are influenced by many unobserved
factors that cannot be captured through aggregated socio-demographic
measures alone. Despite these constraints, the model's relatively low
root mean square error (RMSE = 1.55 crashes per 1,000 residents) and
mean absolute error (MAE = 0.84 crashes per 1,000 residents) suggest
that it has successfully captured consistent, broad patterns in crash
risk that align with socio-economic disparities and urban traffic
dynamics.

From a social risk modeling perspective, these results are significant.
The SHAP analysis and multi-tree plot reveal that the most influential
predictors are not strictly transportation variables but rather
socio-economic indicators that shape exposure and vulnerability. The
post-pandemic indicator emerges as the single strongest factor,
reflecting the profound shift in traffic patterns since 2020, when less
congestion but higher average speeds upended previous risk models by
changing nearly all traffic conditions, especially in densely populated
urban areas like NYC.

Housing and economic variables also feature prominently. Median gross
rent is positively correlated with crash rates, suggesting that denser,
higher-cost urban areas within NYC, with more vehicle-pedestrian
interactions and complex traffic flows, have elevated risk levels.
Similarly, labor force participation peaks as a predictor around
60--70\%, reflecting that areas with higher commuter activity experience
more traffic and thus more collisions.

One of the most striking findings is the role of poverty rate and
vehicle ownership interaction, which appears consistently in the
multi-tree plot. Areas with both high poverty rates and high vehicle
availability show notably higher crash risks, likely due to a
combination of older vehicles, reduced access to safety infrastructure,
and potentially riskier driving environments.

While the model demonstrates that neighborhood-level demographic and
economic factors can act as strong proxies for crash risk, it also
highlights critical limitations for deployment in insurance pricing. The
moderate R² indicates that a significant portion of risk remains
unaccounted for, largely due to unobserved individual-level factors such
as driver age, prior claims, and vehicle safety features. Moreover, the
use of variables like income, race, or foreign-born population---though
statistically predictive---would be highly problematic if directly
incorporated into premium calculations due to both legal prohibitions
and ethical concerns. These variables risk proxy discrimination, where
protected classes are indirectly penalized through correlated
socio-economic attributes.

The model answers the primary research question-what socio-economic
variables are most strongly predictive of auto insurance risk. It shows
that the environmental and demographic facotrs can explain spatial
patterns of crashes and provide insurers or policymakers with
macro-level risk insights. However, the model cannot, and should not, be
used as-is for individual risk assessment. It is better suited for
regional portfolio analysis, identifying high-risk neighborhoods for
targeted safety interventions, or supplementing traditional actuarial
models rather than replacing them.

\section{Conclusions and Future Work}\label{conclusions-and-future-work}

This research demonstrates that gradient boosting, combined with
socio-economic and crash data, can provide interpretable, data-driven
insights into auto insurance risk patterns in NYC. By identifying key
drivers of crash risk---such as post-pandemic shifts, commuting
intensity, and socio-economic vulnerability---the model offers a
foundation for both urban safety planning and high-level risk
assessment.

The study's findings also underscore the limitations of using
aggregated, public data for underwriting. While the model captures broad
trends, it lacks the precision, granularity, and fairness safeguards
required for production-level insurance applications. Therefore, the
model is best suited as a complementary tool for insurers, providing
neighborhood-level risk insights or supporting reinsurance and portfolio
management rather than individual pricing.

Future research should focus on three fronts: (1) integrating behavioral
data, such as telematics, and weather patterns-among other possible
additions-to bridge the gap between macro-level socio-economic patterns
and micro-level driving behavior; (2) developing fairness-aware modeling
approaches to mitigate bias from socio-economic proxies; and (3)
exploring temporal extensions that incorporate evolving risk factors,
including post-pandemic traffic patterns and climate-related hazards.
These directions will help transition from descriptive social risk
modeling to actionable, ethically sound insurance applications.

\section{Appendix A: Variables
Modeled}\label{appendix-a-variables-modeled}

\begin{longtable}[]{@{}
  >{\raggedright\arraybackslash}p{(\linewidth - 6\tabcolsep) * \real{0.1210}}
  >{\raggedright\arraybackslash}p{(\linewidth - 6\tabcolsep) * \real{0.2984}}
  >{\raggedright\arraybackslash}p{(\linewidth - 6\tabcolsep) * \real{0.4516}}
  >{\raggedright\arraybackslash}p{(\linewidth - 6\tabcolsep) * \real{0.1290}}@{}}
\caption{Key variables, descriptions, and transformations in the final
dataset.}\tabularnewline
\toprule\noalign{}
\begin{minipage}[b]{\linewidth}\raggedright
Variable
\end{minipage} & \begin{minipage}[b]{\linewidth}\raggedright
Description
\end{minipage} & \begin{minipage}[b]{\linewidth}\raggedright
Type
\end{minipage} & \begin{minipage}[b]{\linewidth}\raggedright
Transformation
\end{minipage} \\
\midrule\noalign{}
\endfirsthead
\toprule\noalign{}
\begin{minipage}[b]{\linewidth}\raggedright
Variable
\end{minipage} & \begin{minipage}[b]{\linewidth}\raggedright
Description
\end{minipage} & \begin{minipage}[b]{\linewidth}\raggedright
Type
\end{minipage} & \begin{minipage}[b]{\linewidth}\raggedright
Transformation
\end{minipage} \\
\midrule\noalign{}
\endhead
\bottomrule\noalign{}
\endlastfoot
Demographic & pct\_male\_population & Men & Percentage \\
Demographic & pct\_white\_population & Identifying as white &
Percentage \\
Demographic & pct\_black\_population & Identifying as black &
Percentage \\
Demographic & pct\_asian\_population & Identifying as Asian &
Percentage \\
Demographic & pct\_hispanic\_population & Identifying as Hispanic/Latino
& Percentage \\
Demographic & pct\_foreign\_born & Foreign-born & Percentage \\
Age & pct\_age\_under\_18 & Under 18 & Percentage \\
Age & pct\_age\_18\_34 & Aged 18-34 & Percentage \\
Age & pct\_age\_35\_64 & Aged 35-64 & Percentage \\
Age & pct\_age\_65\_plus & Aged 65 and above & Percentage \\
Income/Poverty & median\_income & Median household income
(inflation-adjusted) & Raw value (USD) \\
Income/Poverty & pct\_income\_under\_25k & Households earning less than
\$25,000 & Percentage \\
Income/Poverty & pct\_income\_25k\_75k & Households earning
\$25,000-\$75,000 & Percentage \\
Income/Poverty & pct\_below\_poverty & Below the poverty line &
Percentage \\
Housing & median\_gross\_rent & Median gross rent (USD) & Raw value
(USD) \\
Housing & pct\_owner\_occupied & Owner-occupied housing units &
Percentage \\
Housing & pct\_renter\_occupied & Renter-occupied housing units &
Percentage \\
Education & pct\_less\_than\_hs & Less than high school education &
Percentage \\
Education & pct\_hs\_diploma & High school diploma & Percentage \\
Education & pct\_some\_college & Some college education & Percentage \\
Education & pct\_associates\_degree & Associate's degree & Percentage \\
Education & pct\_bachelors\_degree & Bachelor's degree & Percentage \\
Education & pct\_graduate\_degree & Graduate or professional degree &
Percentage \\
Employment & pct\_in\_labor\_force & In the labor force & Percentage \\
Employment & unemployment\_rate & Unemployment rate & Percentage \\
Transport & pct\_commute\_short & Commute under 15 minutes &
Percentage \\
Transport & pct\_commute\_medium & Commute between 15-30 minutes &
Percentage \\
Transport & pct\_commute\_long & Commute longer than 30 minutes &
Percentage \\
Transport & pct\_carpool & By carpool & Percentage \\
Transport & pct\_public\_transit & By public transit & Percentage \\
Transport & pct\_walk & By walking & Percentage \\
Transport & pct\_bike & By biking & Percentage \\
Transport & pct\_work\_from\_home & Working from home & Percentage \\
Transport & pct\_vehicle & Owns a vehicle & Percentage \\
Engineered & post\_pandemic & Post-pandemic indicator (1 = 2020 and
later) & Binary \\
Engineered & poverty\_vehicle & & \\
\_interaction & Interaction term: poverty rate × vehicle ownership &
Interaction & \\
Engineered & unemployment\_vehicle & & \\
\_interaction & Interaction term: unemployment rate × vehicle ownership
& Interaction & \\
Year & year2018 & Year dummy: 2018 & Indicator \\
Year & year2019 & Year dummy: 2019 & Indicator \\
Year & year2020 & Year dummy: 2020 & Indicator \\
Year & year2021 & Year dummy: 2021 & Indicator \\
Year & year2022 & Year dummy: 2022 & Indicator \\
Year & year2023 & Year dummy: 2023 & Indicator \\
\end{longtable}


\renewcommand\refname{References}
\bibliography{bibliography.bib}



\end{document}
